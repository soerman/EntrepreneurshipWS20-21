\documentclass[10pt,a4paper,noendnumber=true]{scrartcl}
%german umlauts and localization
\usepackage[utf8]{inputenc} 
\usepackage[T1]{fontenc}
\usepackage[ngerman]{babel}
\usepackage[dvipsnames]{xcolor}
\usepackage{booktabs}
\usepackage{tabularx}

\usepackage{geometry}
\geometry{
	a4paper,
	total={170mm,257mm},
	left=20mm,
	top=20mm,
}

%figures etc.
\usepackage[pdftex]{graphicx}
\usepackage{standalone} %externalize files for faster compilation
\usepackage{float}

%mathematical symbols
\usepackage{latexsym} % special symbols 
\usepackage{amsmath,amssymb,amsthm}
\usepackage{textcomp} % supports the Text Companion fonts, which provide many text symbols (such as baht, bullet, copyright, musicalnote, onequarter, section, and yen)
%fonts:
%\usepackage{txfonts}  %supplies virtual text roman fonts using Adobe Times % needs to be loaded AFTER amsmath (because otherwise \iint is defined twice)
\usepackage{mathrsfs}  % for script-like fonts in math mode
\usepackage{nicefrac} % nice fracs in text

%\usepackage{libertine}
%\usepackage[libertine]{newtxmath}
%\usepackage[sc]{mathpazo}
%\linespread{1.05}

%tables
\usepackage{tabularx}

%units
\usepackage{siunitx}

%tikz
\usepackage{tikz}
\usepackage{tikzscale}
\usepackage{pgfplots} 
\usepackage{pgfgantt}
\usepackage{pdflscape}
\usepackage[european]{circuitikz}
\pgfplotsset{compat=newest} 
\pgfplotsset{plot coordinates/math parser=false}
\usetikzlibrary{shapes.geometric}
\usetikzlibrary{arrows.meta}
\usetikzlibrary{arrows}
\usetikzlibrary{shapes.symbols,shadows}

%bib stuff
\usepackage[draft = false]{hyperref}
\usepackage{csquotes}
\usepackage[backend=biber,style=ieee]{biblatex}
%\addbibresource{bib.bib}

% gescheiter Abstand nach paragraph
\newcommand{\properparagraph}[1]{\paragraph{\textcolor{Emerald}{#1}}\mbox{}\\}
\usepackage[parfill]{parskip}

% für die Auflistung von Vor- und Nachteilen in itemize-Umgebung
\newcommand\pro{\item[$+$]}
\newcommand\con{\item[$-$]}

%nice row vector
\newcommand{\rvect}[1]{\begin{bmatrix} #1 \end{bmatrix}}

%align multi pgfplots
\pgfplotsset{yticklabel style={text width=3em,align=right}}

\usetikzlibrary{external}
\tikzexternalize[optimize=false,prefix=tikz/] % activate!

\usepackage{subfig}
\renewcommand{\arraystretch}{1.6}

\title{Entrepreneurship}
\subtitle{Question Catalogue}
\author{}


\begin{document}
\maketitle

\section{Question Catalogue}
\subsection{Session 1}
\properparagraph{What is the activity of an entrepreneur according to Jean Baptiste Say?}
"`The entrepreneur shifts economic resources out of an area of lower and into an area of higher productivity."'

\properparagraph{How does Joseph Schumpeter define “entrepreneurship”?}
Entrepreneurship is about new factor combinations
leading to new products, production methods or new
markets. It is about creative destruction.

\properparagraph{What are the career reasons of nascent entrepreneurs? Name and explain them.}
\begin{itemize}
	\item Independence (freedom \& control in use of one’s time)
	\item Self-realization (pursue self-directed goals)
	\item Financial success (earn money, achieve financial security)
	\item Recognition (have status \& approval by others)
	\item Role (follow family traditions or follow example of others)
	\item Innovation (accomplish something new)
\end{itemize}

\properparagraph{What are the motives for starting a business, according to GEM 2019?}
\begin{itemize}
	\item To make a difference in the world;
	\item To build great wealth or very high income;
	\item To continue a family tradition;
	\item To earn a living because jobs are scarce.
\end{itemize}

\properparagraph{Please name the listed five forms of capital and name at least one example.}
\begin{itemize}
	\item \textbf{Natural Capital} (raw materials, energy, land)
	\item \textbf{Financial Capital} (money, bounds)
	\item \textbf{Intellectual Capital} (human capital: skills, knowledge)
	\item \textbf{Organizational Capital} (machines, patents)
	\item \textbf{Social Capital} (stakeholder relationships)
\end{itemize}

\properparagraph{Please give one example of “Creative Destruction” related to entrepreneurship.}
\begin{itemize}
	\item static: Efficiency and price are main drivers of competition
	\item dynamic: Old established business models declining and failing	
\end{itemize}

\properparagraph{What are the key activities of entrepreneurs according to Byers et al. (2011)?}
%TODO

\properparagraph{What is a Business Opportunity according to Dorf \& Byers?}
An opportunity is a timely and favorable juncture of circumstances providing a good chance for a successful venture.

\properparagraph{Please name the nine categories of an opportunity (mentioned in the lecture).}
\begin{itemize}
	\item Increasing the value of a product or service
	\item New applications of existing means or technologies
	\item Creating mass markets
	\item Customization for individuals
	\item Increasing reach
	\item Managing the supply chain
	\item Convergence of industries
	\item Process innovation
	\item Increasing the scale of the firm
\end{itemize}

\properparagraph{What are the characteristics of an attractive opportunity? Name and explain them shortly.}
\begin{itemize}
	\item Timely: a current need
	\item Solvable: can be solved in the near future with accessible resources
	\item Important: on the priority list of the customer
	\item Profitable: adequate willingness to pay
	\item Context: favorable regulatory and industry situation
	
\end{itemize}

\properparagraph{What is an entrepreneurial competence according to Tittel \& Terzidis? What does competence consist of? Explain the different parts.}
We define a competence as the disposition to generate adequate actions to responsibly solve problems in variable situations.
\begin{itemize}
	\item Knowledge is the body of facts, principles, theories and 	practices that is related to a field (inspired by EQF).
	\item  Skill is the ability to apply knowledge and use know-how to complete tasks (inspired by EQF).
	\item  Attitude is a disposition to respond favorably or unfavorably to an object, person, institution, or event (Ajzen, I. 2005, p.3).
\end{itemize}


\properparagraph{Name the six verbs that describe personal competence according to Tittel \& Terzidis.}
Be, Learn, Create, Want, Dare, Do

\properparagraph{What are the stages of Action Learning?}
\begin{itemize}
	\item Listen: Impulse
	\item Do: Activity \& experience
	\item Reflect: what, how, why, how else
	\item Internalize: Theorize \& practice
\end{itemize}





\newpage
\subsection{Session 2}
\properparagraph{How is strategy defined and what does strategic management deal with according to Terzidis (2017)?}
A strategy is a plan or roadmap for the fundamental behavior of an organization in order to achieve its mission and long-term goals.

Strategic management deals with the initiatives taken by entrepreneurs involving the utilization of resources to enhance the performance of firms in their external environments.

\properparagraph{Name the three key principles for strategic positioning (Porter, 2008).}
\begin{itemize}
	\item Strategy is the creation of a unique and valuable position, involving a different set of activities.
	\item Strategy requires you to make	trade-offs in competing—to choose what not to do.
	\item Strategy involves creating “fit” (coherence) among a company’s activities.
\end{itemize}

\properparagraph{Which questions you need to ask yourself for finding the balance (IKIGAI)?}
\begin{itemize}
	\item What you love,
	\item What you are good at,
	\item What the world needs, and
	\item What you can be paid for.
\end{itemize}

\properparagraph{What is the definition of a firm according Coase (1937)?}
A firm [...] consists of the system of relationships which comes into existence when the direction of resources is dependent on an entrepreneur.

\properparagraph{“The firm as a system” – in which markets is the firm embedded?}
\begin{itemize}
	\item Customer market,
	\item Labor market,
	\item Capital market,
	\item Supplier market,
	\item Complementary partners
\end{itemize}

\properparagraph{What is an industry according to Byers and Dorf?}
An industry is a group of firms producing products that are close substitutes for each other and serve the same customer.

\properparagraph{Please describe the Industry Lifecycle and explain, if necessary, what is meant by the different phases.}
\begin{itemize}
	\item Emerging phase
	\item Growth phase
	\item Mature phase
	\item Declining phase
\end{itemize}


\properparagraph{Please describe Porter’s Five Forces Model – name all of the forces and differentiate between horizontal and vertical competition.}
\begin{itemize}
	\item Horizontal competition:
	\begin{itemize}
		\item Aspiring entrants can require new investment for you to stay competitive.
		\item You must stand up to existing competition.
		\item Substitute offerings can lure customers away.
	\end{itemize}
	\item Vertical competition:
	\begin{itemize}
		\item Customers can force down prices by playing you and your rivals against one another.
		\item Powerful suppliers may constrain your profits if they charge higher prices.	
	\end{itemize}
\end{itemize}

\properparagraph{Give a definition of the term ‘business model’ according to Terzidis (2017).}
A business model is an idealized and aggregated representation of how a firm creates value for all its stakeholders.


\properparagraph{What are the business model V4 dimensions according to Al Debeiand Avison (2011)?}
\begin{itemize}
	\item Value Proposition (business logic of creating value for customers, ...)
	\item Value Architecture (technological architecture and organizational infrastructure)
	\item Value Network (coordination and collaboration)
	\item Value Finance
\end{itemize}


\properparagraph{What are the Business Models Functions (according to Al Debei \& Avison (2011))?}
\begin{itemize}
	\item Alignment Instrument
	\item Interceding Framework
	\item Knowledge Capital
\end{itemize}


\properparagraph{Please name the building blocks of the Business Model Canvas.}
\begin{itemize}
	\item Key partners
	\item Key activities
	\item Key resources
	\item Value proposition
	\item Customer relationships
	\item Channels
	\item Customer segments
	\item Cost structure
	\item Revenue streams
\end{itemize}


\properparagraph{What is a model according to Stachoviak?}
\begin{itemize}
	\item Mapping (model represents something)
	\item Idealization (abstraction, not everything mapped)
	\item Pragmatism (concrete, pragmatic contexts)
\end{itemize}

\properparagraph{Name the limitations of the Business Model Canvas.}
\begin{itemize}
	\item The purpose of the firm is not mentioned.
	\item Market structure and competitors are not mentioned.
	\item There is no explicit mentioning of the ‘value network’.
	\item Implication about the capital needs (e.g. need for working capital) are not mapped.
\end{itemize}

\properparagraph{Fill the business model canvas for Campusjäger!}
%TODO

\properparagraph{Name the ten rules for good Business Design.}
\begin{enumerate}
	\item  Good business model design depends as much on art and intuition as it does on science
	and analysis.
	\item Good business model design requires deep knowledge of customer needs and the
	technological and organizational resources that might meet those needs.
	\item All good business models require an understanding of current business models at work in
	the market. Most new business model designs involve the hybridizations of others.
	\item Alignment and coherence is desirable so that the business model elements will be
	mutually reinforcing.
	\item Strategic analysis must be tied to business model design and vice versa. Strategy guides
	business model design and is also to some extent shaped by it.
	\item Business models should be coupled with strategies and assets that make imitation
	difficult. Imitation will occur sooner or later, and pioneers must be fast learners.
	\item Identifying the customer segment(s) to focus on first in order to learn and achieve
	proof-of-concept and business model viability is a critical capability.
	\item When n-sided markets are involved, getting started early and effectively seeding the
	n sides is critical.
	\item Good business model reengineering skills are an important component of strong
	dynamic capabilities. They enable proficient seizing.
	\item The introduction of new business models into an existing organization is always
	difficult and may require a separate organizational unit.	
\end{enumerate}


\properparagraph{What is a Start-Up \& what is a Lean Start-Up according to Ries?}
\begin{itemize}
	\item \textbf{Start-up:} A startup is a human institution designed to create a new product or service under conditions of extreme uncertainty.
	\item \textbf{Lean Start-up:} Lean Startup is a method to systematically reduce the risk of	projects with high uncertainty.
\end{itemize}

\properparagraph{What are the six stages of validated learning according to Ries?}
\begin{enumerate}
	\item \textbf{IDEAS:} Hypothesis
	\item \textbf{BUILD:} Build experiment
	\item \textbf{PRODUCT:} Minimum viable product
	\item \textbf{MEASURE:} Run experiment, record data
	\item \textbf{DATA:} Collected data
	\item \textbf{LEARN:} Interpret data
\end{enumerate}

\properparagraph{What is a Pivot according to Ries? Give an example and explain shortly.}
A Pivot is a structured course correction designed to test a new fundamental hypothesis about the product, strategy, and engine of growth.






\newpage
\subsection{Session 3}
\properparagraph{Please name 5 top reasons why startups fail according to the lecture.}
\properparagraph{Please give a definition of market, the actors who determine the market according to Homburg	(2017).}
\properparagraph{Give a definition of marketing according to the American Marketing Association.}
\properparagraph{ Name the 4 P’s of the Marketing Mix according to Hisrich and Ramadani (2017). List at least three examples, according to the Lecture.}
\properparagraph{Name the three Key Challenges of New Ventures and the explanation for each.}
\properparagraph{ What is market segmentation according to Wendell Smith?}
\properparagraph{Name the different categories of and list the three mentioned characteristics.}
\properparagraph{How is Value and the Value proposition defined in the lecture (inspired by Byers et al)}
\properparagraph{What are the five dimensions of value of an offering according to Dorf \& Byers?}
\properparagraph{ Please give the definition of a ‘job’ and its three characteristics according to the Job to be done theory (Ulwick 2016).}
\properparagraph{What is the desired outcome? Please rephrase the three listed descriptions of the lecture}
\properparagraph{What are the three approaches to determine the price of an offering?}
\properparagraph{What are the three steps of the Strategic Marketing Process?}
\properparagraph{What are the five stages of innovation diffusion according to Roger (1983) and Moore (1991)?}
\properparagraph{What are the four questions asked in the method described by van Westendorp (1976) What are the key statements of those?}
\properparagraph{Name the 8 listed methods according to the lecture to evaluate the market size.}
\properparagraph{What are the intended outcomes of a competitor analysis according to the lecture? Name the	three listed outcomes.}
\properparagraph{What are the six steps of the competitor analysis? Describe them briefly.}

\newpage
\subsection{Session 4}
\properparagraph{What is a patent (definition used in the lecture)?}
\properparagraph{Name the three highlighted characteristics of a patent and the respective descriptions.}
\properparagraph{What are the findings of patents according to Häussner et al. (2012)?}
\properparagraph{What are the three steps of a patent application?}
\properparagraph{Please name the different steps of the patent application process and assign it to the respective category. Also name all subitems with at least one description per category.}
\properparagraph{What is excluded from patenting in Germany? Name three examples.}
\properparagraph{Please name three opportunities and three risks related to patenting.}
\properparagraph{What is the PCT? What does it provide? Please define as written in the lecture.}
\properparagraph{What is a NTBF? Please define according to the lecture.}
\properparagraph{Please name the Success Factors of New Ventures according to Song et al. (2008).}
\properparagraph{ What is technology push according to the lecture?}
\properparagraph{What is market pull according to the lecture?}
\properparagraph{Name the four levels of the TAS framework according to Terzidis \& Vogel (2017).}
\properparagraph{ What are the 9 Technology Readiness Levels (TRLs) according to the model of NASA?}
\properparagraph{What is a technology according to the lecture? Name all four subitems.}
\properparagraph{Please fill a Technology Canvas on the next page with the historic example of cloud computing given in the lecture.}

%TODO table

\newpage
\subsection{Session 5}
\properparagraph{How defines Terzidis (2019) Leadership and the leader, according to the lecture?}
\properparagraph{ According to Luhmann, what are the three characteristics of a Modern Organization?}
\properparagraph{What are the six traits, that are related to leadership (Northouse, 2016)?}
\properparagraph{What are the five factor personality model dimensions that are related to leadership (Northouse, 2016)?}
\properparagraph{What is the Basic Assumption of Transformal Leadership according to Northouse, 2016)?}
\properparagraph{Please name the six principles of effective leadership according to Malik (2006) and explain	each one in your own words.}
\properparagraph{What are the five tasks of effective leadership according to Malik (2006)?}
\properparagraph{Please define the term ‘decision’ using the definitions of Mintzberg (1976) and the Gabler business dictionary (Wirtschaftslexikon).}
\properparagraph{Please name the 7 stages of a decision-making process according to Malik (2011).}
\properparagraph{What are the seven tools of effective leadership according to Malik?}

\newpage
\subsection{Session 6}
\properparagraph{Name the three forms of external financing mentioned in the lecture. For each of those, name the three outlined definitions.}
\properparagraph{Please name all categories of internal financing for funding a start-up}
\properparagraph{What is the definition of risk (according to the Gabler Online Wirtschaftslexikon)?}
\properparagraph{What is the legal definition of Bankruptcy? Please name the main liability of the management,	according to the Insolvency code.}
\properparagraph{What are the three reasons for insolvency (InsO §17-19)?}
\properparagraph{The risk of the investors depends on the type of investment: Please categorize the different types of external financing (diagram 1). Name the two	definitions for each type.}
\properparagraph{Please fill out the Start-up Financing Cycle (diagram 2).}
\properparagraph{Early-Stages: Why is it difficult for start-ups to uptain loans according to Hof (2017)?	Name all the mentioned aspects.}
\properparagraph{What are the characteristics of the Idea phase according to the lecture?}
\properparagraph{What are the characteristics of the start-up phase according to the lecture?}
\properparagraph{What are the top five motivations of Business Angels for Investing according to Brandenburger et al. (2012)?}
\properparagraph{ Name the five top factors in the category Entrepreneur and team according to Brandenburger et al. (2012).}
\properparagraph{Name the six main phases of the funding process.}
\properparagraph{How do investors identify new companies according the lecture? Name three.}
\properparagraph{Name and explain the several events an investor can make an exit through according to Cumming and MacIntosh (2003).}
\properparagraph{How is a failure defined according to Gage (2012)?}
\properparagraph{Name the 5 top reasons why Startups fail.}

\newpage
\subsection{Session 7}
\properparagraph{Explain the three mentioned reasons why it is necessary to have a business plan according to the lecture.}
\properparagraph{What are the four key objectives of EXIST according to the lecture?}
\properparagraph{Please name a typical structure of a business plan.}
\properparagraph{What are the four sections of the EXIST business plan template? (Gründerstipendium)}
\properparagraph{What are the three definitions mentioned in the lecture for pricing (Simon, 2004; Piercy et al.,	2010; Ingenbleek et al, 2003)?}
\properparagraph{Please explain GbR, GmbH and UG according to the slides in the lecture.}
\properparagraph{What is the definition of Risk according to Vaughan (2008).}
\properparagraph{Please explain the Van Westendorp-Method using your own words and point out the four main questions}
\properparagraph{Please name the five points of the Consequence Scale including the explanations.}
\properparagraph{Please name the five points of the Likelihood Scale including the explanations.}
\properparagraph{Please name the three options to treat risks including the explanation, mentioned in the	lecture.}
\properparagraph{Please name the steps of the Linear causation approach.}
\properparagraph{Please name the steps of the Linear causation approach in an entrepreneurial context.}
\properparagraph{Please name and explain the five principle of effectuation according to Sarasvathy (2009)?}

\end{document}

